%% 
%% Copyright 2007, 2008, 2009 Elsevier Ltd
%% 
%% This file is part of the 'Elsarticle Bundle'.
%% ---------------------------------------------
%% 
%% It may be distributed under the conditions of the LaTeX Project Public
%% License, either version 1.2 of this license or (at your option) any
%% later version.  The latest version of this license is in
%%    http://www.latex-project.org/lppl.txt
%% and version 1.2 or later is part of all distributions of LaTeX
%% version 1999/12/01 or later.
%% 
%% The list of all files belonging to the 'Elsarticle Bundle' is
%% given in the file `manifest.txt'.
%% 
%% Template article for Elsevier's document class `elsarticle'
%% with harvard style bibliographic references
%% SP 2008/03/01

\documentclass[preprint,12pt,authoryear]{elsarticle}

%% Use the option review to obtain double line spacing
%% \documentclass[authoryear,preprint,review,12pt]{elsarticle}

%% Use the options 1p,twocolumn; 3p; 3p,twocolumn; 5p; or 5p,twocolumn
%% for a journal layout:
%% \documentclass[final,1p,times,authoryear]{elsarticle}
%% \documentclass[final,1p,times,twocolumn,authoryear]{elsarticle}
%% \documentclass[final,3p,times,authoryear]{elsarticle}
%% \documentclass[final,3p,times,twocolumn,authoryear]{elsarticle}
%% \documentclass[final,5p,times,authoryear]{elsarticle}
%% \documentclass[final,5p,times,twocolumn,authoryear]{elsarticle}

%% For including figures, graphicx.sty has been loaded in
%% elsarticle.cls. If you prefer to use the old commands
%% please give \usepackage{epsfig}

%% The amssymb package provides various useful mathematical symbols
\usepackage{amssymb}
%% The amsthm package provides extended theorem environments
\usepackage{amsthm}

%% The lineno packages adds line numbers. Start line numbering with
%% \begin{linenumbers}, end it with \end{linenumbers}. Or switch it on
%% for the whole article with \linenumbers.
%% \usepackage{lineno}


\usepackage{bm}
\usepackage{amsmath}
%\usepackage{natbib}
 \usepackage{lipsum}


\theoremstyle{cupthm}
\newtheorem{thm}{Theorem}[section]
\newtheorem{prop}[thm]{Proposition}
\newtheorem{cor}[thm]{Corollary}
\newtheorem{lemma}[thm]{Lemma}
\theoremstyle{cupdefn}
\newtheorem{defn}[thm]{Definition}
\theoremstyle{cuprem}
\newtheorem{rem}[thm]{Remark}
\numberwithin{equation}{section}


\journal{Applied Mathematics Letters}

\begin{document}

\begin{frontmatter}

%% Title, authors and addresses

%% use the tnoteref command within \title for footnotes;
%% use the tnotetext command for theassociated footnote;
%% use the fnref command within \author or \address for footnotes;
%% use the fntext command for theassociated footnote;
%% use the corref command within \author for corresponding author footnotes;
%% use the cortext command for theassociated footnote;
%% use the ead command for the email address,
%% and the form \ead[url] for the home page:
%% \title{Title\tnoteref{label1}}
%% \tnotetext[label1]{}
%% \author{Name\corref{cor1}\fnref{label2}}
%% \ead{email address}
%% \ead[url]{home page}
%% \fntext[label2]{}
%% \cortext[cor1]{}
%% \address{Address\fnref{label3}}
%% \fntext[label3]{}

\title{Rank Minimization for Sylvester Equation}

%% use optional labels to link authors explicitly to addresses:
%% \author[label1,label2]{}
%% \address[label1]{}
%% \address[label2]{}

\author{}

\address{}

\begin{abstract}
Sylvester equation is widely used in many problems in system and automatic control. In this paper, we prove that the well known Roth's similarity theorem (\cite{5}) is a special case of a rank minimization theorem for Sylvester equation.
\end{abstract}

\begin{keyword}
Rank Minimization, Sylvester Equation

%% PACS codes here, in the form: \PACS code \sep code

%% MSC codes here, in the form: \MSC code \sep code
%% or \MSC[2008] code \sep code (2000 is the default)

\end{keyword}

\end{frontmatter}

%% \linenumbers

%% main text

%% The Appendices part is started with the command \appendix;
%% appendix sections are then done as normal sections
%% \appendix

\section{Introduction}
The Sylvester matrix equation is widely used in system and automatic control community \cite{1,2,3,4,5}. In \cite{1}, Dehghan and Hajarian propose two algorithms for finding Hermitian reflexive and skew-Hermitian solutions of Sylvester matrix equations. In \cite{2}, Hajarian proposes a gradient based iterative method to solve the Sylvester matrix equation. In \cite{3}, Wu \textsl{et al.} solves the Sylvester matrix equatio via Kronecker map. In \cite{4}, Hu and Cheng proposed a polynomial solution to the Sylvester matrix equation. Denote by $\bm{A}\in\mathbb{F}^{m\times m}$, $\bm{B}\in\mathbb{F}^{n\times n}$, and $\bm{C}\in\mathbb{F}^{m\times n}$ three given matrices over some field $\mathbb{F}$. Denote by $\text{GL}(n,\mathbb{F})=\{\bm{M}\in\mathbb{F}^{n\times n}|\text{det}\bm{M}\neq0\}$ the general linear group of degree $n$ over the field $\mathbb{F}$. The following theorem is the well-known Roth's similarity theorem, which provides a sufficient and necessary condition for guaranting the consistency of the Sylvester equation (\ref{e1}) in terms of an equivalence between two associated matrices.



\begin{thm}[\cite{5}] The matrix equation 
\begin{equation}
\label{e1}
\bm{A}\bm{X}-\bm{X}\bm{B}=\bm{C}
\end{equation}
is solvable w.r.t $\bm{X}\in\mathbb{F}^{m\times n}$ if and only if there exists a matrix $\bm{P}\in\text{GL}(m+n,\mathbb{F})$ such that 
\[
\bm{P}
\left(
\begin{array}{cc}
\bm{A} & \bm{C} \\
\bm{0} & \bm{B}
\end{array}
\right)
=
\left(
\begin{array}{cc}
\bm{A} & \bm{0} \\
\bm{0} & \bm{B}
\end{array}
\right)
\bm{P}
\]
\end{thm}

\section{Main Results}
We now generalize the Roth's similarity theorem to a general result based on rank minimization. The purpose of this note is to prove the following theorem.
\begin{thm} Given $\bm{A},\bm{B},\bm{C}$ defined above, denote by we have
\\
\centerline{
$
\min
\{
\textnormal{rank}
(\bm{A}\bm{X}-\bm{X}\bm{B}-\bm{C})
|
\bm{X}\in\mathbb{F}^{m\times n}
\}
$
}
\[
=
\min
\left\{
\textnormal{rank}
\left[
\bm{P}
\left(
\begin{array}{cc}
\bm{A} & \bm{C} \\
\bm{0} & \bm{B}
\end{array}
\right)
-
\left(
\begin{array}{cc}
\bm{A} & \bm{0} \\
\bm{0} & \bm{B}
\end{array}
\right)
\bm{P}
\right]
|
\bm{P}
\in
\textnormal{GL}(m+n,\mathbb{F})
\right\}
.
\]
\end{thm}

\begin{proof}
Denote by 
\begin{equation}
\alpha(\bm{X})=\bm{A}\bm{X}-\bm{X}\bm{B}-\bm{C},
\end{equation}
and
\begin{equation}
\beta(\bm{P})=
\bm{P}
\left(
\begin{array}{cc}
\bm{A} & \bm{C} \\
\bm{0} & \bm{B}
\end{array}
\right)
-
\left(
\begin{array}{cc}
\bm{A} & \bm{0} \\
\bm{0} & \bm{B}
\end{array}
\right)
\bm{P}.
\end{equation}
Define 
\begin{equation}
R_{\alpha}
=
\min\textnormal{rank}
\{
\alpha(\bm{X})
|
\bm{X}
\in
\mathbb{F}^{m\times n}
\}
\end{equation}
and 
\begin{equation}
R_{\beta}
=
\min\textnormal{rank}
\{
\beta(\bm{P})
|
\bm{P}
\in
\textnormal{GL}(m+n,\mathbb{F}).
\}
\end{equation}
From \cite{6,7}, we can obtain that
\begin{equation}
R_{\alpha}
=
\textnormal{rank}
\left(
\begin{array}{cc}
\bm{A} & \bm{C} \\
\bm{0} & \bm{B}
\end{array}
\right)
-
\textnormal{rank}
\left(
\begin{array}{cc}
\bm{A} & \bm{0} \\
\bm{0} & \bm{B}
\end{array}
\right).
\end{equation}
For $\bm{P}\in\textnormal{GL}(m+n,\mathbb{F})$, we can obtain that
\begin{equation}
\begin{split}
\textnormal{rank}
\beta(\bm{P})
&
\ge
\textnormal{rank}
\bm{P}
\left(
\begin{array}{cc}
\bm{A} & \bm{C} \\
\bm{0} & \bm{B}
\end{array}
\right)
-
\textnormal{rank}
\left(
\begin{array}{cc}
\bm{A} & \bm{0} \\
\bm{0} & \bm{B}
\end{array}
\right)
\bm{P}
\\
&
=
\textnormal{rank}
\left(
\begin{array}{cc}
\bm{A} & \bm{C} \\
\bm{0} & \bm{B}
\end{array}
\right)
-
\textnormal{rank}
\left(
\begin{array}{cc}
\bm{A} & \bm{0} \\
\bm{0} & \bm{B}
\end{array}
\right)
\\
&
=
R_{\alpha}.
\end{split}
\end{equation}
Hence, we have that $R_{\beta} \ge R_{\alpha}$.
On the other hand, we denote by 
\begin{equation}
\bm{P}_{\bm{X}}
=
\left(
\begin{array}{cc}
\bm{I} & \bm{X} \\
\bm{0} & \bm{I}
\end{array}
\right),
\end{equation}
then we have 
\begin{equation}
\beta(\bm{P}_{\bm{X}})
=
\left(
\begin{array}{cc}
\bm{0} & -\alpha(\bm{X}) \\
\bm{0} & \bm{0}
\end{array}
\right).
\end{equation}
Therefore, 
\begin{equation}
R_{\beta}
\le
\min
\{
\textnormal{rank}
\beta(\bm{P}_{\bm{X}})
|
\bm{X}
\in
\mathbb{F}^{m\times n}
\}
=
\min
\{
\textnormal{rank}
\alpha(\bm{X})
\}
=
R_{\alpha}.
\end{equation}
This completes the proof.
\end{proof}







%% If you have bibdatabase file and want bibtex to generate the
%% bibitems, please use
%%
%%  \bibliographystyle{elsarticle-harv} 
%%  \bibliography{<your bibdatabase>}

%% else use the following coding to input the bibitems directly in the
%% TeX file.

\section*{Reference}

\begin{thebibliography}{100}

%% \bibitem[Author(year)]{label}
%% Text of bibliographic item


\bibitem[Dehghan(2011)]{1} Mehdi Dehghan, Masoud Hajarian, Two algorithms for finding the Hermitian reflexive and skew-Hermitian solutions of Sylvester matrix equations, Applied Mathematics Letters, Volume 24, Issue 4, Pages 444-449, 2011.

\bibitem[Hajarian(2016)]{2} Masoud Hajarian, Solving the general Sylvester discrete-time periodic matrix equations via the gradient based iterative method, Applied Mathematics Letters, Volume 52, Pages 87-95, 2016.

\bibitem[Wu(2008)]{3} Ai-Guo Wu, Feng Zhu, Guang-Ren Duan, Ying Zhang, Solving the generalized Sylvester matrix equation AV+BW=EVF via a Kronecker map, Applied Mathematics Letters, Volume 21, Issue 10, Pages 1069-1073, 2008.

\bibitem[Hu(2006)]{4} Qingxi Hu, Daizhan Cheng, The polynomial solution to the Sylvester matrix equation, Applied Mathematics Letters, Volume 19, Issue 9, Pages 859-864, 2006.

\bibitem[Roth(1952)]{5} R. E. Roth, `The equations $AX-BY=C$ and $AX-XB=C$ in equations', \textsl{Proc. Amer. Math. Soc.} \textbf{3}, 392-396, 1952.

\bibitem[Tian(2002)]{6} Y. Tian, `The minimal rank of matrix expression $A-BX-YC$', \textsl{Missouri J. Math. Sci.} \textbf{14}, 40-48, 2002.

\bibitem[TianMT(2000)]{7} Y. Tian, `Rank equalities related to generalized inverses of matrices and their applications', Master Thesis, Montreal, Quebec, Canada, 2000.

\end{thebibliography}


\end{document}

\endinput
%%
%% End of file `elsarticle-template-harv.tex'.
