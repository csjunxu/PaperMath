%% This file is a template for the preparation of articles for
%% submission to the Bulletin of the Australian Mathematical Society.
%% Version: 2010-12-07 for baustms.cls v2.1 DET
\documentclass{baustms}
%% The default behaviour of the document class 'baustms' is to colour
%% some words (such as section headings, Theorem, Proof) and to establish
%% hyperlinks from citations to the entries in the bibliography.
%% The option 'plain' turns off hyperlinks and colour.
%\documentclass[plain]{baustms}

%% - packages
%% The class file loads the following packages
%% 1) The AmS-LaTeX packages: amsfonts, amssymb, amsmath, amscd, amsthm
%% 2) txfonts, graphicx, color, enumerate

%% - tables
%% If your article uses tables you are strongly advised to include the
%% 'booktabs' package by uncommenting the line below.  Refer to the
%% package documentation for further instructions.
%\usepackage{booktabs}

%% - citesort
%% The \citesort command loads the hypernat package and the natbib
%% package with option 'sort&compress'.  This means that citations
%% of the form [3,2,1] will be compressed to [1-3]. Furthermore the `1'
%% and the `3' will be coloured and linked to the  bibliography.
\citesort

%% - operator names
%% Declare all mathematics operators here, for example
%\DeclareMathOperator{\Hom}{Hom}

%% - theorems etc.
%% The following commands number theorems, lemmas etc 
%% in a single sequence, within sections, using the 
%% Cambridge University Press (Bulletin) style



\theoremstyle{cupthm}
\newtheorem{thm}{Theorem}[section]
\newtheorem{prop}[thm]{Proposition}
\newtheorem{cor}[thm]{Corollary}
\newtheorem{lemma}[thm]{Lemma}
\theoremstyle{cupdefn}
\newtheorem{defn}[thm]{Definition}
\theoremstyle{cuprem}
\newtheorem{rem}[thm]{Remark}
\numberwithin{equation}{section}
%\newtheorem{conj}[thm]{Conjecture}
%\newtheorem{quest}[thm]{Question}
%\newtheorem{example}[thm]{Example}





\usepackage{bm}

\begin{document}
\runningtitle{Short title for running head (top of right hand page)}
\title{Rank Minimization for the Generalized Sylvester Equation}


%% If there is more than one author, put \cauthor immediately before
%% the corresponding author.
%\cauthor %% mark the next author as corresponding author
\author[1]{Jun Xu}
\address[1]{Department of Computing, The Hong Kong Polytechnic University, Hung Hom, Hong Kong, China \email{csjunxu@comp.polyu.edu.hk}}
%% If there are several authors, list them here
%\author[2]{Second author}
%\address[2]{Second address\email{a@net.com}}

%% List the authors, initials and surnames only, for the
%% running head (left hand page)
\authorheadline{Jun Xu}

%% If there is a dedication, include it here
%\dedication{Dedicated to ...}

\support{Include acknowledgement of support here}

\begin{abstract}
The well known Roth's similarity theorem is a special case of a rank minimization theorem. 
\end{abstract}

%% - subject classification and keywords
%% 2010 American Mathematical Society Subject Classification
%% Provide only ONE primary classification
\classification{primary 15A24}
%% Four or five keywords or phrases
\keywords{Roth's similarity theorem, rank minimization, Sylvester equation}

\maketitle

\section{Introduction}
Denote by $\bm{A}\in\mathbb{F}^{m\times m}$, $\bm{B}\in\mathbb{F}^{n\times n}$, and $\bm{C}\in\mathbb{F}^{m\times n}$ three given matrices over some field $\mathbb{F}$. Denote by $\text{GL}(n,\mathbb{F})=\{\bm{M}\in\mathbb{F}^{n\times n}|\text{det}\bm{M}\neq0\}$ the general linear group of degree $n$ over the field $\mathbb{F}$. The following theorem is the well-known Roth's similarity theorem, which provides a sufficient and necessary condition for guaranting the consistency of the Sylvester equation (\ref{e1}) in terms of an equivalence between two associated matrices.

% For alignments use AmS-LaTeX constructions not \eqnarray.

%% - theorems and proofs
\begin{thm}[\cite{Roth}] The matrix equation 
\begin{equation}
\label{e1}
\bm{A}\bm{X}-\bm{X}\bm{B}=\bm{C}
\end{equation}
is solvable w.r.t $\bm{X}\in\mathbb{F}^{m\times n}$ if and only if there exists a matrix $\bm{P}\in\text{GL}(m+n,\mathbb{F})$ such that 
\[
\bm{P}
\left(
\begin{array}{cc}
\bm{A} & \bm{C} \\
\bm{0} & \bm{B}
\end{array}
\right)
=
\left(
\begin{array}{cc}
\bm{A} & \bm{0} \\
\bm{0} & \bm{B}
\end{array}
\right)
\bm{P}
\]
\end{thm}
We now generalize the Roth's similarity theorem to a general result based on rank minimization. The purpose of this note is to prove the following theorem.
\begin{thm} Given $\bm{A},\bm{B},\bm{C}$ defined above, denote by we have
\\
\centerline{
$
\min
\{
\textnormal{rank}
(\bm{A}\bm{X}-\bm{X}\bm{B}-\bm{C})
|
\bm{X}\in\mathbb{F}^{m\times n}
\}
$
}
\[
=
\min
\left\{
\textnormal{rank}
\left[
\bm{P}
\left(
\begin{array}{cc}
\bm{A} & \bm{C} \\
\bm{0} & \bm{B}
\end{array}
\right)
-
\left(
\begin{array}{cc}
\bm{A} & \bm{0} \\
\bm{0} & \bm{B}
\end{array}
\right)
\bm{P}
\right]
|
\bm{P}
\in
\textnormal{GL}(m+n,\mathbb{F})
\right\}
.
\]
\end{thm}

\begin{proof}
Denote by 
\begin{equation}
\alpha(\bm{X})=\bm{A}\bm{X}-\bm{X}\bm{B}-\bm{C},
\end{equation}
and
\begin{equation}
\beta(\bm{P})=
\bm{P}
\left(
\begin{array}{cc}
\bm{A} & \bm{C} \\
\bm{0} & \bm{B}
\end{array}
\right)
-
\left(
\begin{array}{cc}
\bm{A} & \bm{0} \\
\bm{0} & \bm{B}
\end{array}
\right)
\bm{P}.
\end{equation}
Define 
\begin{equation}
R_{\alpha}
=
\min\textnormal{rank}
\{
\alpha(\bm{X})
|
\bm{X}
\in
\mathbb{F}^{m\times n}
\}
\end{equation}
and 
\begin{equation}
R_{\beta}
=
\min\textnormal{rank}
\{
\beta(\bm{P})
|
\bm{P}
\in
\textnormal{GL}(m+n,\mathbb{F}).
\}
\end{equation}
From \cite{Tian,TianMT}, we can obtain that
\begin{equation}
R_{\alpha}
=
\textnormal{rank}
\left(
\begin{array}{cc}
\bm{A} & \bm{C} \\
\bm{0} & \bm{B}
\end{array}
\right)
-
\textnormal{rank}
\left(
\begin{array}{cc}
\bm{A} & \bm{0} \\
\bm{0} & \bm{B}
\end{array}
\right).
\end{equation}
For $\bm{P}\in\textnormal{GL}(m+n,\mathbb{F})$, we can obtain that
\begin{equation}
\begin{split}
\textnormal{rank}
\beta(\bm{P})
&
\ge
\textnormal{rank}
\bm{P}
\left(
\begin{array}{cc}
\bm{A} & \bm{C} \\
\bm{0} & \bm{B}
\end{array}
\right)
-
\textnormal{rank}
\left(
\begin{array}{cc}
\bm{A} & \bm{0} \\
\bm{0} & \bm{B}
\end{array}
\right)
\bm{P}
\\
&
=
\textnormal{rank}
\left(
\begin{array}{cc}
\bm{A} & \bm{C} \\
\bm{0} & \bm{B}
\end{array}
\right)
-
\textnormal{rank}
\left(
\begin{array}{cc}
\bm{A} & \bm{0} \\
\bm{0} & \bm{B}
\end{array}
\right)
\\
&
=
R_{\alpha}.
\end{split}
\end{equation}
Hence, we have that $R_{\beta} \ge R_{\alpha}$.
On the other hand, we denote by 
\begin{equation}
\bm{P}_{\bm{X}}
=
\left(
\begin{array}{cc}
\bm{I} & \bm{X} \\
\bm{0} & \bm{I}
\end{array}
\right),
\end{equation}
then we have 
\begin{equation}
\beta(\bm{P}_{\bm{X}})
=
\left(
\begin{array}{cc}
\bm{0} & -\alpha(\bm{X}) \\
\bm{0} & \bm{0}
\end{array}
\right).
\end{equation}
Therefore, 
\begin{equation}
R_{\beta}
\le
\min
\{
\textnormal{rank}
\beta(\bm{P}_{\bm{X}})
|
\bm{X}
\in
\mathbb{F}^{m\times n}
\}
=
\min
\{
\textnormal{rank}
\alpha(\bm{X})
\}
=
R_{\alpha}.
\end{equation}
This completes the proof.
\end{proof}

% An end-of-proof symbol (open box) will be typeset at the
% end of the proof.



% \ack % or \acks
% Put acknowledgements here


\begin{thebibliography}{99}
\bibitem{Roth} R. E. Roth, `The equations $AX-BY=C$ and $AX-XB=C$ in equations', \textsl{Proc. Amer. Math. Soc.} \textbf{3} (1952), 392-396.

\bibitem{Tian} Y. Tian, `The minimal rank of matrix expression $A-BX-YC$', \textsl{Missouri J. Math. Sci.} \textbf{14} (2002), 40-48.

\bibitem{TianMT} Y. Tian, `Rank equalities related to generalized inverses of matrices and their applications', Master Thesis, Montreal, Quebec, Canada (2000).
\end{thebibliography}

% alteratively, bibliographies prepared with BibTeX can be included by
% means of the following commands
%\bibliographystyle{srtnumbered}
%\bibliography{mybib}


\end{document}

%% *Other constructions*

%% - enumerations
%% The 'enumerate' package is loaded by the class file and
%% therefore the following constructions are available.
To typeset a list of items number (i), (ii), ... use
\begin{enumerate}[(i)]
\item first item
\item second item
\end{enumerate}

To typeset a list of items number H1, H2, ... use
\begin{enumerate}[H1]
\item first item
\item second item
\end{enumerate}


%% - figures and tables}
\begin{figure}
\centering
%\includegraphics{x.eps}
\caption{Caption text}\label{fig1:}
\end{figure}

%% The following example assumes that 'booktabs' package is loaded
\begin{table}
\caption{Caption text}\label{tab:1}
\centering
\begin{tabular}{cccc}
\toprule
\multicolumn{2}{c}{text} & \multicolumn{2}{c}{text}\\
\cmidrule(r){1-2}\cmidrule(l){3-4}
\multicolumn{1}{c}{One} & Two & \multicolumn{1}{c}{Three} & Four\\
\midrule
1 & 2 & 3 & 4 \\
1 & 2 & 3 & 4 \\
\bottomrule
\end{tabular}
\end{table}
